\documentclass{article}
\usepackage{graphicx, amsmath,svg, float} % Required for inserting images
\usepackage[ngerman]{babel}

\title{Ergebnisse Schwinger-Modell}
\author{Florian Stein}
\date{\today}

\begin{document}
	\maketitle

\section{Masseloser Fall}
\subsection{Grundzustandsenergie}
 
\begin{figure}[H]
\centering
\hspace*{-0cm}\includesvg[width=.8\linewidth]{/home/itg/Documents/LGT/Rechnungen/Bachelorarbeit/Oitmaa-paper/daten/Grundzustand.svg}
\caption{Grundzustand für Werte von $Nag$ zwischen 5 und 20, sowie Gittergrößen $N$ zwischen 2 und 24. Zusätzlich Fitpolynome vom Grad 8 zur Extrapolation ins Kontinuum. Der Schwarze Punkt entspricht dem exakten Wert.}
\end{figure}
Für größeres $Nag$ liegen die Kurven recht nah beieinander, das endliche Volumen sollte also keinen sehr großen Einfluss haben.

Für $Nag=20$ reicht die Rechenleistung nicht mehr wirklich um nah genug ans Kontinuum zu kommen.

Die extrapolierten Werte hängen mehr oder weniger Stark vom Grad des Fitpolynoms ab, ab einem Grad von $\approx$ 5 sehen die Werte einigermaßen gescheit aus.

Die exrapolierten Werte für $Nag=5,10$ bzw. $Nag=15,20$ schwanken dabei ungefähr in einem Bereich von 0,002  bzw. 0,02 um den Theoriewert.
\subsection{Vektormasse}

\begin{figure}[H]
	\centering
	\hspace*{-0cm}\includesvg[width=1\linewidth]{/home/itg/Documents/LGT/Rechnungen/Bachelorarbeit/Oitmaa-paper/daten/M1.svg}
	\caption{Analog für die Vektormasse. Fitpolynome vom Grad 7}
\end{figure}

Hier weichen die verschiedenen Kurven schon deutlich stärker voneinander ab. 
Die Extrapolationswerte schwanken um ca. 0,1, je nach Grad des Polynoms.
\end{document}
